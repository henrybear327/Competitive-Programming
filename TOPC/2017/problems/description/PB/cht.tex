\begin{center}
    {\LARGE Problem B}\vspace{1mm}\\
    {\Large The Combination of Poker Cards}\\
    {Time limit: 1 second}\\
    {Memory limit: 512 megabytes}
\end{center}

\textbf{\large Problem Description}

撲克牌是一種國際盛行的卡牌遊戲,有很多種玩法,據傳最早源自18世紀。
標準撲克牌共52張牌,包含四種花色,每一花色共13張不同點數的牌。在華人
地區盛行的撲克牌玩法有大老二和十三支,主要是比較牌型組合的大小。如果
想要利用電腦設計這些撲克牌遊戲,就必須寫程式自動判斷各種牌型。假設我
們簡化牌型的判斷條件如下: 在不看花色與只看牌點的條件下,每次從給定的6
張牌中,決定牌型是哪一種。如果使用整數數字代表不同的牌點,則可能的牌
型描述如下:
\begin{itemize}
\item 單張(single): 6個數字均不相同,例如,2 5 7 10 9 8
\item 一對(one pair): 兩個數字相同,其餘均不同,例如,4 4 7 10 8 9
\item 兩對(two pairs): 兩組兩個數字相同,例如,8 8 3 3 6 7
\item 三對(three pairs): 三組兩個數字相同,例如,8 8 3 3 7 7
\item 三條(one triple): 一組三個數字相同,其餘均不同,例如,2 2 2 7 5 6
\item 雙三條(two triples): 兩組三個數字相同,例如,2 2 2 7 7 7
\item 鐵支(tiki): 四個數字相同,例如,5 5 5 5 9 8
\item 鐵支配對(tiki pair): 四個數字相同,另外還有一對數字,例如,5 5 5 5 9
9
\item 葫蘆(full house): 三個數字相同,另外還有一對數字,例如,3 3 3 9 9 7
%\item 順(straight): 5個連號的數字,例如,4 5 6 7 8 10
\end{itemize}
假設使用整數數字1至13代表不同點數的牌卡,請寫一個程式從每行輸入的6個數字中判讀是以上哪一種牌型。

\textbf{\large Input Format}

第一行輸入包含一個整數$T$,表示測試的牌型數量。每一個測試牌型包括6個
落於1至13的整數數字,數字間用單一空格區隔。

可假設:
\begin{itemize}
    \tightlist{}
    \item $1 \le T \le 25$
    \item 不會有同一數字出現超過四次。
\end{itemize}

\textbf{\large Output Format}

輸出每一行的牌型名稱,牌型名稱請用對應的英文名稱,即single、one
pair、two pairs、three pairs、one triple、two triples、tiki、tiki pair或full house。

\textbf{\large Sample Input}

\begin{verbatim}
5
4 3 4 3 12 10
5 4 2 3 6 12
10 12 10 12 12 8
8 5 8 8 5 5
2 10 6 10 10 10
\end{verbatim}

\textbf{\large Sample Output}
\begin{verbatim}
two pairs
single
full house
two triples
tiki
\end{verbatim}
