\begin{center}
    {\LARGE Problem D}\vspace{1mm}\\
    {\Large Mixing Coins}\\
    {Time limit: 5 seconds}\\
    {Memory limit: 512 megabytes}
\end{center}

\textbf{\large Problem Description}

御坂喜歡用硬幣當作電磁炮射擊。

為了打擊犯罪,她準備了一排硬幣。為了生產出更強力的硬幣,她會把硬幣們混合在一起。然而,不同材質
的硬幣之間並不相容,所以她只會將相同材質的硬幣混合。

以下是御坂製作硬幣的步驟:

\begin{enumerate}
    \item 從序列的頭開始,找出第一組連續三個相同材質的硬幣
    \item 將它們從序列取出
    \item 混合在一起,生產出一枚新的相同材質的硬幣
    \item 將新硬幣放回序列尾端
\end{enumerate}

她會重複做這些步驟,直到她不能再生產新的硬幣。

御坂想要知道她最後會有多少硬幣。請趕快幫她計算硬幣吧!

\textbf{\large Input Format}

第一行有一個整數 $T$ ,表示測試資料的數量。

每組測試資料的第⼀⾏有一個正整數 $N$,表⽰御坂有 $N$ 組連續的硬幣。所有硬幣都在一排之中。

接著有 $N$ 行,每一行有一個字元 $c_i$ 和正整數 $n_i$ ,表示第 $i$ 組有連續 $n_i$ 個材質
為 $c_i$ 的硬幣,接在第 $(i-1)$ 組硬幣之後。

\begin{itemize}
    \tightlist{}
    \item $T \leq 10$
    \item $1 \leq N \leq 10^5$
    \item $1 \leq n_i \leq 10^9$
    \item $c_i$ 是一個英文大寫字母,$c_i \neq c_{i+1}$ 對於 $1 \leq i < N$
\end{itemize}

\textbf{\large Output Format}

對於每組測試資料,輸出一個整數於一行,表示御坂在做完盡量多次製作硬幣的步驟之後,有多少硬幣。

\textbf{\large Sample Input}

\begin{verbatim}
2
3
A 3
B 1
A 2
3
A 2
B 3
A 2
\end{verbatim}

\textbf{\large Sample Output}
\begin{verbatim}
2
3
\end{verbatim}

